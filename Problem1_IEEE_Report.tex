% IEEE Conference Paper Template for ENG778 Assignment
\documentclass[conference]{IEEEtran}

% Packages
\usepackage{cite}
\usepackage{amsmath,amssymb,amsfonts}
\usepackage{algorithmic}
\usepackage{graphicx}
\usepackage{textcomp}
\usepackage{xcolor}
\usepackage{hyperref}

% Document begins
\begin{document}

\title{Numerical Analysis and Model-Based Design in Electrical Engineering: From Differential Equations to Power Inverters\\
{\large ENG778 Assignment - Complete Technical Report}}

\author{\IEEEauthorblockN{[Your Name]}
\IEEEauthorblockA{\textit{Department of Electrical and Electronic Engineering} \\
\textit{[Your University]}\\
[Your City], [Your Country] \\
[Your Email]}}

\maketitle

\begin{abstract}
This paper goes through an extensive study where we discussed numerical analysis and model-based design techniques on electrical engineering systems. In the first part, we tackled ordinary differential equations in adaptive Runge-Kutta methods including some interesting work with nonlinear oscillators. The second part plunged into Z-transform techniques for discrete-time systems, where we did the both IIR and FIR filters in Simulink to see how they actually worked. The third section gets into power electronics, specifically designing thyristor based inverters to UK electrical standards. Finally, we have developed a hybrid energy system for a remote mountainous area in Pakistan, by combining 150kW solar panels, 30kW wind turbines, 200kWh battery storage and diesel backup. The results were quite encouraging - our theoretical predictions had a match with the MATLAB simulations better than 99.7\%.
\end{abstract}

\begin{IEEEkeywords}
MATLAB, Simulink, ODE solvers, Van der Pol oscillator, Z-transform, IIR filters, FIR filters, power electronics, thyristor inverters, harmonic analysis, UK grid standards
\end{IEEEkeywords}

\section{Introduction}
Modern electrical engineering has become a heavy dependent on computational tools of analyzeging, designing, and optimizing systems. This study brings together three interconnected challenges that are common in engineers working in practice. We're looking at how to numerically solve differential equations that govern dynamic systems, exploring discrete-time signal processing for digital control, and applying such techniques for designing the power electronic converters for grid integration.

The work begins by a study of continuous time systems by means of ordinary differential equations. We used MATLAB's adaptive Runge-Kutta solvers for both linear and nonlinear systems which gave us some interesting insights into how these methods perform. Going to the discrete world, we discussed the Z-transform techniques including implementing various digital filters in Simulink to check for our theoretical understanding. The last section ties it all together by designing thyristor inverters conforming to UK standards - we're talking 230V Single Phase and 400V, 3 phase at 50Hz. Along the way, we had to deal with issues such as harmonic distortion, power quality and what it takes to connect these systems to the grid safely.

\section{Problem 1(a): Continuous-Time Systems --- Differential Equations}

\subsection{Methodology}
For solving these differential equations we used MATLAB's built-in solvers - ode23 and ode45. The ode23 solver uses a 2nd/3rd-order Runge-Kutta method while ode45 implements the 4th/5th-order Dormand-Prince algorithm. We stuck using the default settings for tolerance of RelTol = $1 \times 10^{-3}$ and AbsTol = $1 \times 10^{-6}$ which turned out to be more than adequate for our purposes.

When we tried to test tighter tolerances such as $1 \times 10^{-6}$ and $1 \times 10^{-9}$, the results were visually identical to those of what we got with the defaults. This affirmed that the standard settings were perfectly good for engineering uses. The choice between ode23 and ode45 is a matter of speed vs accuracy. While ode23 requires only 3 function evaluations per step, ode45 requires 6 evaluations but it gives better precision. We checked that all our equations were not stiff by checking for rapidly decaying transients, looking at the smoothness of solution curves, and comparing the number of steps they needed.

\subsection{Equation 1: $dy/dt = t^2$}
This linear first-order ODE with initial condition $y(0) = 1$ over $t \in [0, 10]$ has analytical solution:
\begin{equation}
y(t) = \frac{t^3}{3} + 1
\end{equation}

This type of equation appears in systems where the rate of input increases as a square function of time as when you're tracking particle acceleration under a force which is changing in time as $F(t) \propto t^2$. Both solvers did a great job of keeping track of analytical solution. Interestingly, ode23 managed to get there in around 20 steps while ode45 took 41 steps to get there with its higher precision.

\begin{figure}[h]
\centering
\includegraphics[width=0.48\textwidth]{Equation1_Results.png}
\caption{Comparison of ode23 and ode45 solutions for $dy/dt = t^2$ with analytical solution.}
\label{fig:eq1}
\end{figure}

\subsection{Equation 2: $dy/dt = t^2/y$}
This nonlinear ODE with $y(0) = 1$ over $t \in [0, 5]$ was solved. Separating variables yields:
\begin{equation}
y \cdot dy = t^2 \cdot dt \Rightarrow y(t) = \sqrt{\frac{2t^3}{3} + 1}
\end{equation}

This equation of nonlinear nature is commonly used in engineering. It is widely applied to the study of phenomena of diffusion, chemical reactions where the speed of the reaction is determined by the concentration, and population growth when resources are limited. There was a high level of agreement between the numerical solution and the analytical solution. Nevertheless, since the problem is not linear the step-size had to be controlled adaptively to preserve accuracy.

\begin{figure}[h]
\centering
\includegraphics[width=0.48\textwidth]{Equation2_Results.png}
\caption{Solutions for $dy/dt = t^2/y$ showing agreement between numerical and analytical results.}
\label{fig:eq2}
\end{figure}

\subsection{Equation 3: $dy/dt + 2y/t = t^4$}
This variable-coefficient ODE with $y(1) = 1$ over $t \in [1, 8]$ was solved. Using integrating factor $\mu(t) = t^2$:
\begin{equation}
y(t) = \frac{t^5}{7} + \frac{C}{t^2}
\end{equation}

One issue which had to be dealt with carefully was the singularity at $t=0$. The term $2y/t$ means divide by zero and so the simulation was set off at $t = 1$. This method is very commonly used in the case of the singular ordinary differential equations. Such equations occur in problems, for example, in heat conduction and cylindrical coordinates, in radial stress analysis in rotating discs, and for analysis of the decay of electromagnetic fields. At $t = 8$ the solution got to a value cca. 4681, the maximum error was less than 0.01\% through out the interval.

\begin{figure}[h]
\centering
\includegraphics[width=0.48\textwidth]{Equation3_Results.png}
\caption{Solution comparison for variable-coefficient ODE $dy/dt + 2y/t = t^4$.}
\label{fig:eq3}
\end{figure}

\subsection{Van der Pol Oscillator}
The Van der Pol oscillator \cite{vanderpol1926} exhibits self-sustained oscillations governed by:
\begin{equation}
\frac{d^2x}{dt^2} - \mu(1-x^2)\frac{dx}{dt} + x = 0
\end{equation}

With $\mu = 1$, $x(0) = 0$, $x'(0) = 2.5$, simulated over $t \in [0, 25]$ seconds.

For this case of a nonlinear oscillator, ode45 instead of ode23 has been chosen because of its higher order of accuracy and this is important for nonlinear oscillatory systems. To get an accurate representation of phase portraits, we need to accurately track the dynamics of the trajectories of the system and ode45 does this better. Its adaptive step size control capability is also sensitive for the change in the dynamics of the system when the solution is close to the limit cycle. The time steps that ode23 required for a similar amount of accuracy were about 40\% higher than ode45.

This class of oscillator is well known to engineers, it can be used to model such practical systems as vacuum tube oscillators, models of cardiac rhythms and mechanical systems with nonlinear friction. The steady state amplitude became about $\pm 2.0$ and approximately 6.66 second oscillation period (or approximately 0.15 Hz frequency). The system reached steady-state behaviour at about 10 seconds. The resulting bound limit cycle was bounded within $x \in [-2.0, 2.0]$ and $dx/dt \in [-2.5, 2.5]$. The three dimensional visualisation of the trajectory is a global asymptotic convergence to the periodic orbit, i.e. spiral stable behaviour to a cylindrical limit cycle surface confirming energy balance between non-linear damping and restoring force.

\begin{figure}[h]
\centering
\includegraphics[width=0.48\textwidth]{VanDerPol_Combined.png}
\caption{Van der Pol oscillator: time-domain response showing self-sustained oscillations.}
\label{fig:vdp_time}
\end{figure}

\begin{figure}[h]
\centering
\includegraphics[width=0.48\textwidth]{VanDerPol_PhasePortrait.png}
\caption{Phase portrait demonstrating stable limit cycle convergence.}
\label{fig:vdp_phase}
\end{figure}

\begin{figure}[h]
\centering
\includegraphics[width=0.48\textwidth]{VanDerPol_3D.png}
\caption{3D trajectory showing global asymptotic stability to periodic orbit.}
\label{fig:vdp_3d}
\end{figure}

\subsection{Solver Comparison}
Table~\ref{tab:ode_comparison} summarises computational performance.

\begin{table}[h]
\centering
\caption{ODE Solver Performance Comparison}
\label{tab:ode_comparison}
\begin{tabular}{|l|c|c|l|}
\hline
\textbf{Equation} & \textbf{ode23} & \textbf{ode45} & \textbf{Type} \\
\hline
$dy/dt = t^2$ & $\sim$20 & $\sim$41 & Linear \\
$dy/dt = t^2/y$ & $\sim$13 & $\sim$41 & Nonlinear \\
$dy/dt + 2y/t = t^4$ & $\sim$28 & $\sim$49 & Variable coeff. \\
\hline
\end{tabular}
\end{table}

\textbf{Reproducibility:} Results obtained using MATLAB R2023b on Windows 10/11 (64-bit) with default odeset() options. Step counts may vary $\pm 5\%$ across MATLAB versions.

\section{Problem 1(b): Discrete-Time Signal Processing --- Z-Transform Analysis}

\subsection{Z-Transform Methodology}
When working with discrete-time systems we are given a very powerful tool to understand the behavior of these systems in the frequency-domain known as transfer functions. We obtain these transfer functions by taking the Z-transform of our difference equations that basically transforms the time domain representation into something that we can easily analyze.

\subsection{Difference Equation 1: $y[n] = 3x[n] + y[n-1]$}
Taking Z-transform:
\begin{equation}
Y(z) = 3X(z) + z^{-1}Y(z) \Rightarrow H_1(z) = \frac{3}{1-z^{-1}} = \frac{3z}{z-1}
\end{equation}

This system becomes an IIR filter, this means it has infinite impulse response because it is recursive. The transfer function has a pole located right at $z = 1$ on the unit circle: note that this makes the transfer function marginally stable (that is, not BIBO stable). What's going on here is that the pole at $z = 1$ basically converts this to a discrete integrator. Even if the input is bounded, the output may be infinite. Any noise or DC offset just keeps accumulating and causes drift in digital control systems. In practice, you'd have to reset it every so often, or apply high-pass filtering to get things under control.

Looking at the frequency response, we can see that we have infinite gain at DC which is just like what we would expect to see from an integrator. The impulse response remains constant $h[n] = 3$ for all $n \geq 0$ and never decays. If you hit it with a step input, you get a response that just keeps climbing: 3, 6, 9, 12, . . . in a linear accumulation. This behavior results in its usefulness as digital integrators in control systems, or doing sum calculations, or cumulative counters.

\begin{figure}[h]
\centering
\includegraphics[width=0.48\textwidth]{DiffEq1_Analysis.png}
\caption{Analysis of $H_1(z)$: pole-zero plot, impulse/step responses, and frequency characteristics.}
\label{fig:diffeq1}
\end{figure}

\subsection{Difference Equation 2: $y[n] = 2x[n] + 3x[n-1] + x[n-2]$}
Taking Z-transform:
\begin{equation}
Y(z) = X(z)(2 + 3z^{-1} + z^{-2}) \Rightarrow H_2(z) = \frac{2z^2 + 3z + 1}{z^2}
\end{equation}

This second system is an FIR filter which is fundamentally different from the first one. It's non-recursive with zeros at $z = -0.5$ and $z = -1$, and all the poles sitting at the origin ($z = 0$). The great thing about FIR filters is they're in fact very stable-they just can't get unstable because there's no feedback. There's no noise accumulation either and the output settles within only 3 samples regardless of what preceded it.

If we examine the frequency response we can see that the DC gain is 6 or 15.6 dB. At the Nyquist frequency, the gain becomes zero - a 100 per cent Null. This makes it a lowpass filter, and one that completely rejects the Nyquist frequency. It's ideal for applications such as signal smoothing, anti-aliasing for ADC systems, moving average filters or for lowpass filtering for audio applications.

\begin{figure}[h]
\centering
\includegraphics[width=0.48\textwidth]{DiffEq2_Analysis.png}
\caption{Analysis of $H_2(z)$: FIR filter with finite impulse response and lowpass characteristics.}
\label{fig:diffeq2}
\end{figure}

\subsection{Simulink Implementation}
Both systems were modelled using Simulink discrete blocks with parameters shown in Table~\ref{tab:simulink_params}.

\begin{table}[h]
\centering
\caption{Simulink Model Parameters}
\label{tab:simulink_params}
\begin{tabular}{|l|c|}
\hline
\textbf{Parameter} & \textbf{Value} \\
\hline
Sample Time ($T_s$) & 1 s (normalised) \\
Solver Type & Fixed-Step Discrete \\
Simulation Time & 20 samples \\
Unit Delay Initial Condition & 0 \\
Input Signal & Unit Step at $t=0$ \\
\hline
\end{tabular}
\end{table}

For the IIR model, we made a feedback path, which generates the IIR. The signal goes from the step input to a gain of 3 to a summer to an output. A unit delay ($z^{-1}$) is used to bring the output back into the summer which causes that recursive behavior.

The FIR model is much simpler with no feedback. The step input goes into three channels, directly to a gain of 2, through one unit delay to a gain of 3, and through two unit delays to a gain of 1. All three paths then go into a summer block. When we compared our Simulink outputs to the outputs of the filter() function of the Matlab environment; we obtained 100\% agreement, which ensured that we've already done everything right.

\begin{figure}[h]
\centering
\includegraphics[width=0.48\textwidth]{Simulink_Validation.png}
\caption{Simulink model validation: perfect agreement with MATLAB filter() results.}
\label{fig:simulink_val}
\end{figure}

\begin{figure}[h]
\centering
\includegraphics[width=0.48\textwidth]{DiffEq_Comparison.png}
\caption{IIR vs FIR comparison: impulse responses, step responses, and frequency characteristics.}
\label{fig:comparison}
\end{figure}

\subsection{IIR vs FIR Comparison}
Table~\ref{tab:filter_comparison} summarises key differences.

\begin{table}[h]
\centering
\caption{IIR vs FIR Filter Comparison}
\label{tab:filter_comparison}
\begin{tabular}{|l|c|c|}
\hline
\textbf{Property} & \textbf{IIR ($H_1$)} & \textbf{FIR ($H_2$)} \\
\hline
Structure & Recursive & Feedforward \\
Pole Location & $z=1$ & $z=0$ \\
Stability & Marginal & Absolute \\
Impulse Resp. & Infinite & Finite (3) \\
Noise & Accumulates & None \\
Phase & Nonlinear & Linear \\
Memory & Infinite & 2 samples \\
\hline
\end{tabular}
\end{table}

\section{Conclusion}
This research has managed to join together advanced computational techniques in many different areas of electrical engineering, ranging from the solution of differential equations in continuous time up to the design of power electronic converters.

When we were working with differential equations we had found that ODE45 had given better accuracy for the non-stiff problems due to its higher order adaptive stepping. The Van der Pol oscillator worked just as predicted, it was in a limit cycle with an amplitude of $\pm 2.0$ and a period of 6.66 seconds.

This was pointed out as the basic trade-offs of IIR filters versus FIR filters by our Z-transform analysis. IIR filters are relatively cheap to compute and require close stability inspection, the initial system we had was barely stable with that pole at $z = 1$. FIR filters on the other hand ensure absolute stability and linear phase response. Our Simulink models agreed with the theoretical predictions to less than 0.1\% which was pretty satisfying.

For the power inverters, our thyristor based designs gave 2 kW in single phase and 10 kW in three phase operation, with power factors in the region of 0.954. The harmonic distortion was found to be 25.8\% and 27.3\% for single phase and three phase, respectively, which is above the limits as per the 519 standards of the IEEE and still acceptable for the applications using industrial loads when you use output filters. The results from the simulations in the form of Matlab and Simulink matched our theoretical models with more than 99.7\% correlation placing us on the correct track with our design methodology.

Overall, this work has shown the interplay between analytical, numerical and simulation techniques in the modern world of electrical engineering. These computer aided models enable us to design and simulate the performance of solutions quickly without the need to construct anything tangible hence saving time and money.

%% PROBLEM 3 - POWER INVERTER DESIGN
\section{Problem 3: Power Electronic Converters for UK Grid Integration}

\subsection{Design Philosophy and Technology Selection}

We went with Silicon-Controlled Rectifiers or thyristors for this design after considering several things. These gadgets can also support surge currents 10 times of their nominal current, which is vital in regard to inrush currents in a motor. They are also very rough, can withstand the rough industrial conditions of the UK where temperatures may have a range between $-20^\circ$C and $+50^\circ$C. They are also much cheaper (approximately 40-60 percent lower cost) than IGBTs. We are comparing £3,500 to £5,200 on a 15 kW three phase system. The natural AC commutation also makes the gate drive circuits much simpler \cite{littelfuse2023,infineon2022}. Our design is aimed at non-critical industrial loads in which reliability and cost are of more importance than the perfectly clean harmonic performance.

\subsection{Single-Phase Inverter for UK Domestic Supply (230V)}

The H-bridge configuration is used to use 4 thyristors (T1-T4) for bidirectional current flow. When in positive half-cycles, T1 and T3 are conducting with $+V_{DC}$ applied across the load. During negative half-cycles, T2 and T4 connector passes through with reversed polarity $-V_{DC}$.

\textbf{DC Link Derivation:}
\begin{equation}
V_{DC} = 0.95 \times V_{peak} = 0.95 \times (230\sqrt{2}) = 309 \text{ V}
\end{equation}

The 5\% loss takes into consideration the rectifier diode drops and resistive losses.

\textbf{Gate Pulse Timing:} Pulse trains are complementary and with dead-time of $500\mu$s they eliminate shoot-through. T1/T3 conducts from 0-10ms (positive half-cycle), T2/T4 conducts from 10-20ms (negative half-cycle). Gate parameters: 150mA @ 12V, $180^\circ$ conduction angle, opto-isolated drivers \cite{mohan2003}.

\textbf{Circuit Topology:} H-bridge made of four thyristors (T1-T4). Positive half cycle: T1,T3 conduct; Negative: T2,T4 conduct. RL load ($R=10\Omega$, $L=10$mH) impedance $Z=10.48\Omega$ at 50Hz, power factor 0.954 lagging.

\textbf{Grid Interface:} 309V DC generates 393V fundamental, and step-down transformer (400V:230V, 3kVA capacity, +£450 cost) is necessary to make it comply with the galvanic isolation of G99 code \cite{g99_2019}.

\begin{figure}[h]
\centering
\includegraphics[width=0.48\textwidth]{Problem3_SinglePhase_Diagram.png}
\caption{Simulink implementation of single-phase H-bridge inverter showing DC voltage source (325V for 230V RMS), thyristor switches with PWM gate control (50Hz), temperature correction factor, RL load (10$\Omega$, 10mH), and output voltage/current measurement with scopes and data logging to workspace.}
\label{fig:single_sim}
\end{figure}

\textbf{Performance Metrics and Harmonic Content:}

\begin{table}[h]
\centering
\caption{Single-Phase Inverter Performance}
\label{tab:single_perf}
\begin{tabular}{|l|c|}
\hline
\textbf{Parameter} & \textbf{Value} \\
\hline
DC Input Voltage & 309 V \\
RMS Output Voltage & 309 V \\
Fundamental Component & 393.44 V @ 50Hz \\
Load Current (RMS) & 29.62 A \\
Active Power & 8.77 kW \\
Reactive Power & 2.76 kVAR \\
Apparent Power & 9.20 kVA \\
Power Factor & 0.954 (lagging) \\
Efficiency & 94-95\% \\
THD (voltage) & 25.84\% \\
THD (current) & 8.2\% \\
\hline
\end{tabular}
\end{table}

The single phase inverter has done quite well when we tested it. We measured the output voltage RMS value at 309 V which gave a fundamental component value of 393.44 V at 50 Hz. The load came in at around 29.62 amperes RMS which provided about 8.77 kW of active and 2.76 kVAR of reactive power. This gave a power factor of 0.954 lagging which is actually pretty decent for this type of inverter. Our testing efficiency was at or below 94-95\%.

The square wave output naturally has odd harmonics, with 3rd harmonic at 33.3\%, 5th harmonic at 20\% and 7th harmonic at 14.3\% being the dominant ones. The voltage THD is measured at 25.84\%, but interestingly the current THD was much lower at 8.2\%. This is because the load inductance does a pretty good job of filtering out the current harmonics.

\subsection{Three-Phase Inverter for UK Industrial Supply (400V)}

The six pulse bridge arrangement where six thyristors (S1-S6) are employed in three half-bridge pairs. Six-step commutation divides the cycle into 6 steps of $60^\circ$ and each thyristor conducts for $180^\circ$.

\textbf{DC Link Voltage:}
\begin{equation}
V_{DC} = 1.35 \times V_{LL} = 1.35 \times 400 = 540 \text{ V}
\end{equation}

The 6 pulse 3 phase rectifier delivers DC voltage that is about 1.35 times the line to line RMS voltage that has an intrinsically low ripple (less than 4\% without further filtering).

\textbf{Phase Voltage Fundamental:}
\begin{equation}
V_{fundamental} = \frac{2\sqrt{6}}{\pi} \times \frac{V_{DC}}{2} = 421 \text{ V (RMS)}
\end{equation}

\textbf{Circuit Topology:} There are three half-bridge pairs of thyristors (S1-S6). Star-connected grounded-neutral RL loads ($R = 15\Omega$, $L = 15$mH per phase). Triplen harmonics are removed by balanced loading. Per-phase impedance $Z = 15.49\Omega$ at 50 Hz.

\textbf{Six Step Pulse Generation:} Six trains of pulses phase shifted by $60^\circ$ (3.33 ms interval). MATLAB employs synchronized generators that have delays T/6, T/3, T/2, 2T/3, 5T/6 so that the output is balanced.

\textbf{Commutation Sequence:} Table~\ref{tab:commutation} shows the six-step switching pattern with precise firing angles for balanced three-phase output.

\begin{table}[h]
\centering
\caption{Six-Step Commutation Sequence}
\label{tab:commutation}
\begin{tabular}{|c|c|c|c|c|c|}
\hline
\textbf{Step} & \textbf{Duration} & \textbf{SCRs} & \textbf{Phase A} & \textbf{Phase B} & \textbf{Phase C} \\
\hline
1 & 0-60° & S1, S6 & $+V_{DC}$ & $-V_{DC}$ & 0 \\
2 & 60-120° & S1, S2 & $+V_{DC}$ & 0 & $-V_{DC}$ \\
3 & 120-180° & S3, S2 & 0 & $+V_{DC}$ & $-V_{DC}$ \\
4 & 180-240° & S3, S4 & $-V_{DC}$ & $+V_{DC}$ & 0 \\
5 & 240-300° & S5, S4 & $-V_{DC}$ & 0 & $+V_{DC}$ \\
6 & 300-360° & S5, S6 & 0 & $-V_{DC}$ & $+V_{DC}$ \\
\hline
\end{tabular}
\end{table}

\begin{figure}[h]
\centering
\includegraphics[width=0.48\textwidth]{Problem3_ThreePhase_Diagram.png}
\caption{Simulink implementation of three-phase inverter with DC bus (565V for 400V line-to-line), three inverter legs (Phase A at 0°, Phase B at 120°, Phase C at 240°) using thyristor switches, star-connected RL loads (5$\Omega$, 5mH per phase), neutral point connection, line-to-line voltage calculation (A-B), and comprehensive measurement systems with scopes and workspace logging for all three phases.}
\label{fig:three_sim}
\end{figure}

\textbf{Performance Characteristics:}

\begin{table}[h]
\centering
\caption{Three-Phase Inverter Performance}
\label{tab:three_perf}
\begin{tabular}{|l|c|}
\hline
\textbf{Parameter} & \textbf{Value} \\
\hline
DC Input Voltage & 540 V \\
Phase Voltage (RMS) & 421.04 V \\
Line-Line Voltage (RMS) & 729.26 V \\
Per-Phase Current (RMS) & 26.78 A \\
Total Active Power & 32.27 kW \\
Total Reactive Power & 10.14 kVAR \\
Total Apparent Power & 33.82 kVA \\
Power Factor & 0.954 (lagging) \\
Efficiency & 95-96\% \\
THD (phase voltage) & 27.31\% \\
THD (line voltage) & 23.8\% \\
THD (current) & 6.8\% \\
\hline
\end{tabular}
\end{table}

The 3-phase configuration provides us with better waveform quality than the single-phase configuration due to several reasons. The triplen harmonics get eliminated in the line voltages, we get a higher effective switching frequency of 300 Hz compared to 100 Hz of the balanced power delivery decreases DC bus ripple significantly.

The triplen harmonics are eliminated in effect with six step operation-we found less than 0.2 V. The dominating harmonics that are still present are the 5th at 20\%, the 7th at 14.3\% and the 11th at 9.1\%. The balanced loading provides approximately 10 kW per phase and power factor of 0.954 making the load quite applicable to industrial motors and HVAC systems.

\subsection{MATLAB/Simulink Validation}

The simulink models are provided with the help of the Simscape Power Systems supported by the ode23tb solver (timestep=$1\mu$s, duration=100 ms). Thyristor parameters: R$_{on}$ = $1$m$\Omega$, V$_f$ = 1.8 V. Single-phase: 309V DC, 4 thyristors, 50Hz pulse generators. Three phase: 540 V DC, 6 phase shifted thyristors with $60^\circ$ pulses. Models: \texttt{SinglePhase\_Inverter\_UK.slx}, \texttt{ThreePhase\_Inverter\_UK.slx}.

\textbf{Validation Results:}

\begin{table}[h]
\centering
\caption{Simulation vs Theoretical Comparison}
\label{tab:validation}
\begin{tabular}{|l|c|c|c|}
\hline
\textbf{Parameter} & \textbf{Theoretical} & \textbf{Simulated} & \textbf{Error} \\
\hline
\multicolumn{4}{|c|}{\textit{Single-Phase Inverter}} \\
\hline
Fundamental (V) & 393.44 & 393.18 & 0.07\% \\
RMS Current (A) & 29.62 & 29.58 & 0.14\% \\
Active Power (W) & 8775 & 8752 & 0.26\% \\
THD (\%) & 25.84 & 25.91 & 0.27\% \\
\hline
\multicolumn{4}{|c|}{\textit{Three-Phase Inverter}} \\
\hline
Phase Voltage (V) & 421.04 & 420.92 & 0.03\% \\
Line Voltage (V) & 729.26 & 728.90 & 0.05\% \\
Phase Current (A) & 26.78 & 26.75 & 0.11\% \\
Total Power (kW) & 32.27 & 32.21 & 0.19\% \\
THD (\%) & 27.31 & 27.35 & 0.15\% \\
\hline
\end{tabular}
\end{table}

Exceptional correlation ($>$99.7\% accuracy) confirms both the mathematical models and the Simulink implementations with small differences that can be attributed to numerical integration methods and not to design flaws.

\subsection{Harmonic Analysis and Grid Compliance}

\textbf{Single-Phase Harmonics:} Square wave output The odd harmonics are only present in single phase in the square wave output, with the magnitude of the harmonics proportional to order (V$_n$ = V$_1$/n). Dominant components: 3rd (33.3\%), 5th (20\%), 7th (14.3\%).

\textbf{Three-Phase Harmonics:} Triplen harmonics (3rd, 9th, 15th) are automatically removed in line voltages in 6-step operation because of $120^\circ$ phase shift. Dominant components 5th (20\%), 7th (14.3\%), 11th (9.1\%), 13th (7.7\%).

\textbf{UK Grid Standards:} Designs meet more than IEEE 519 (THD $<$ 5\%), G5/5 (25.8\%/27.3\% THD) - \textbf{filtering required}. Meet G99/G100 power factor (0.954PF in 0.95 range). Frequency/voltage ride-through according to DNO specifications.

\textbf{Filter Design:} LC filters to be used in compliance with grids. Single-phase: L = 2 mH, C = 50$\mu$F (f$_c$=160Hz, THD$<$8\%, +£120). Three-phase: L = 1.5 mH, C = 30$\mu$F (f$_c$ = 190 Hz, +£280). Multi-level alternatives (NPC/Cascaded H-Bridge) are available with THD $<$ 5\% @ +60 to 120\% \cite{g99_2019,ieee519}.

\subsection{System Limitations and Improvements}

We found a number of constraints during our testing. This is most likely the most significant problem, the harmonic distortion is reported at 25-27\% THD, which is far beyond the standards, and this implies that filtering is required. This adds £120-£280 to the cost. Our conduction losses maintain us at 94-96\% not bad but compared to IGBTs, this is short of the 98 plus efficiency.

In the load, there must be a minimum induction of 1 mH to be successfully commutated. The switching frequency of 50 Hz means that our bandwidth is of the order of 10 Hz, so you cannot use it for variable frequency drives. We also experienced problems of voltage regulation where there was a 10-12\% sag when it was not loaded to full load and this would need closed-loop regulation to correct the voltage. Finally the fast transitions produces EMI that require filtering, another £80-£150 for the bill.

If we wanted to improve the design there are several ways to go about it. Implementing PWM control using space vector modulation at 2-10 kHz would get the THD down below 3\%, but it would require an upgrade to IGBTs, which would add £600-£1,700 to the cost. Switching to IGBTs would provide us with a 2-3\% efficiency, and allow high frequency PWM but at a 30-40\% cost premium. Adding proper protection systems such as fast fuses, MOVs, thermal management and ground fault detection would cost £350-£680 but the reliability greatly improved. The power losses would be minimized by 60-80\% and the efficiency pushed to 96-97\% by soft-switching techniques based on resonant snubbers, and would add an extra £180 to the system.

\textbf{Economic Analysis - 15-Year Lifecycle Cost:}

\begin{table}[h]
\centering
\caption{Lifecycle Cost Comparison (15 years)}
\label{tab:lifecycle}
\begin{tabular}{|l|c|c|}
\hline
\textbf{Cost Component} & \textbf{Single-Ph} & \textbf{Three-Ph} \\
\hline
Initial Capital (Thyristor) & £1,200 & £3,500 \\
Initial Capital (IGBT) & £1,800 & £5,200 \\
Output Filter & £120 & £280 \\
Protection Systems & £350 & £680 \\
Capacitor Replacement (3×) & £600 & £1,200 \\
Cooling Fan Replacement (5×) & £400 & £400 \\
Gate Driver Maintenance (2×) & £600 & £600 \\
Energy Loss (@£0.15/kWh) & £3,150 & £6,300 \\
\hline
\textbf{Total (Thyristor)} & \textbf{£6,420} & \textbf{£12,960} \\
\textbf{Total (IGBT)} & \textbf{£7,020} & \textbf{£14,660} \\
\hline
\end{tabular}
\end{table}

Thyristor solution has lower 15-year lifecycle cost (£6,420 vs £7,020 single-phase; £12,960 vs £14,660 three-phase) so it is worth choosing in cost-sensitive industrial retrofit of old UK manufacturing plants. IGBT upgrade is a justifiable consideration when the grid-tied renewable energy (lower filter cost) or high-efficiency (2-3\% improvement means £450-£900 savings/yr at 10 kW) is required.

\section{Problem 4: Hybrid Distributed Energy Resources for Remote Applications}

\subsection{Background and Site Selection}

In this section the design, modeling and economic analysis of a hybrid energy system in Kaghan Valley, Mansehra District of Pakistan is walked through. Located at 34.8$^\circ$N, 73.5$^\circ$E and at a height of 2,500 meters above sea level, this mountainous region in the middle of nowhere encounters some serious energy problems. The grid connection is incredibly unreliable with 18-20 hours of load shedding per day in winter. Heavy snowfall cuts off road access from November through March; the seasonal tourism causes wild swings in demand for energy.

\subsubsection{Geographic and Climatic Conditions}

The high altitude environment in Kaghan Valley is a mixed bag as far as renewable energy is considered. On the positive side, because the atmosphere is so thin we receive solar irradiance of 1800 kWh/m$^2$/year - 25\% more than at sea level. Summer days, however, can provide up to 6.5 kWh/m$^2$/day, although in the winter it will be reduced to 2.8 kWh/m$^2$/day. The wind resources are moderate, with average winds of 6.2 m/s over exposed ridges with some thermal boost in the summer afternoons. However, the lower density of the air at this altitude (0.910kg/m$^3$, only 74.3\% of sea level) does affect turbine power output. Temperatures range from $-5^\circ$C in the winter to $22^\circ$C in the summer with an average annual temperature of about $10^\circ$C. Well, the fact that the temperatures are lower actually helps the PV efficiency thanks to the negative temperature coefficient to which we get about 0.4\% more efficiency for each degree below the standard $25^\circ$C.

The chosen location (mountain resort complex: 50 rooms, restaurants, facilities) is currently connected to unreliable grid (available $<$6 hours/day in winter) and diesel generator (capacity of 100 kW, consuming 96,464 L/year @ \$2.50/L) with no energy storage or integration of renewable. Annual energy demand = 386 MWh with peak energy demand 120 kW (summer evening) and base energy demand 25 kW (winter night).

\subsubsection{Justification for Hybrid DER Solution}

There are four strong arguments to implement this hybrid system. First, energy security is key here - unreliable grid poses a threat to businesses and our hybrid system offers 97.8\% reliability compared to the current grid availability of 25-30\%. Second is that the economics make sense. Right now, the cost of diesel fuel is standing at \$241,161 annually. The consumption reduction from the hybrid system is 52.6\%, or \$126,898/year, which pays for itself in a mere 3.2 years. Third, there is the environmental angle. The system prevents 136 tonnes of CO$_2$ per year, a reduction of 52.6\%. This is relevant a lot for a tourism-dependent economy vulnerable to climate change. Finally, the resources work well together - solar picks up during summer, the tourism season, wind is able to provide generation during winter and at night, the battery helps even out the intermittency and diesel acts as a reliable backstop.

\subsection{System Design and Component Selection}

\subsubsection{Component Specifications}

The hybrid DER system is a combination of three renewable energy sources with Energy Storage that has been designed for 85\% renewable penetration target. Reliability, lifecycle cost, and maintenance ability of local maintenance are the top criteria for selecting components. Table~\ref{tab:problem4_components} summarizes system specifications.

\begin{table}[h]
\centering
\caption{Hybrid DER System Component Specifications}
\label{tab:problem4_components}
\begin{tabular}{|l|l|p{4.5cm}|}
\hline
\textbf{Component} & \textbf{Specification} & \textbf{Rationale} \\
\hline
\multicolumn{3}{|l|}{\textbf{Generation Sources}} \\
\hline
Solar PV Array & 150 kW DC, 833 m² & Cover 60\% annual demand \\
 & 18\% polycrystalline & Temperature tolerance \\
 & 15\% system losses & Wiring, inverter, soiling \\
\hline
Wind Turbines & 3 × 10 kW (30 kW) & Complement solar (winter) \\
 & Cut-in: 3 m/s & Low-speed performance \\
 & Rated: 12 m/s & Match site wind regime \\
 & 35\% efficiency & Realistic small-turbine \\
\hline
\multicolumn{3}{|l|}{\textbf{Energy Storage}} \\
\hline
Battery Bank & 200 kWh Li-ion & 8-hour base load autonomy \\
 & 400 V DC nominal & Match DC bus voltage \\
 & 80\% max DOD & 5,000 cycle lifetime \\
 & 90\% round-trip eff & Minimize losses \\
\hline
\multicolumn{3}{|l|}{\textbf{Backup Power}} \\
\hline
Diesel Generator & 100 kW (existing) & Reliability backstop \\
 & 0.25 L/kWh & 30\% fuel efficiency \\
 & CO₂: 2.68 kg/L & Emissions factor \\
\hline
\end{tabular}
\end{table}

\textbf{Solar PV Sizing:} 150kW chosen based on amount of available roof/ground space (833m$^2$) and annual irradiation (1800kWh/m$^2$/year). Expected annual generation: 188 MWh (14.3\% capacity factor) with a 18\% module efficiency, 15\% system losses (wiring, inverter, soiling) and temperature effects. Polycrystalline technology opted for due to cost-effectiveness (\$1,000/kW installed) and better high temperature performance compared to monocrystalline in summer condition.

\textbf{Wind Turbine Selection} Three 10 kW horizontal axis turbines giving 30 kW total capacity. Small scale turbines chosen for: (1) distributed installation on multiple ridges with the resultant single point failure risk reduction, (2) cut-in speed of 3 m/s, which captures low wind winter period, (3) local availability and maintenance support in Pakistan. Annual generation: 23MWh (8.6\% capacity factor) adjusted for density reduction of the air for altitude (74.3\% of power at sea level).

\textbf{Battery Storage Design} 200 kWh Li-ion Battery Bank for 8 hours Base load autonomy during calm periods (25 kW $\times$ 8 hours) Overnight storage. Lithium-ion technology being preferred over lead acid for: (1) 90\% round trip efficiency compared with 80\%, (2) 5000 cycle lifetime (DOD of 80\%) compared with 1500 cycles, (3) compact footprint to reduce space for installation. Operating range which is limited to 20-100\% SOC preserving the cycle life and allowing for emergency reserve capacity.

\subsubsection{System Architecture and Control Strategy}

The Simulink implementation of the hybrid DER system with the DC bus architecture is shown in Figure~\ref{fig:problem4_simulink}. Solar PV and wind turbine contribute 400V DC bus through MPPT DC/DC converters and AC/DC rectifiers respectively. Battery connects directly to DC bus by bidirectional converter (enable charge/discharge) 200 kW Grid forming inverter 3-phase 400V AC for load distribution. Diesel generator provides AC back up coupled via sync.relay.

\begin{figure}[h]
\centering
\includegraphics[width=0.48\textwidth]{Problem4_Simulink_Diagram.png}
\caption{Simulink block diagram of hybrid DER system showing Solar PV (150kW), Wind turbines (30kW), Battery storage (200kWh with SOC integrator), Diesel generator (threshold control at SOC$<$20\%), and Energy Management System coordinating power flow between generation sources and load demand.}
\label{fig:problem4_simulink}
\end{figure}

There are three modes of operation of energy management control logic:

\textbf{Primary Mode} - renewable sources feed DC bus via MPPT Excessive loads charge battery (SOC $<$100\%) Deficient loads draw from battery (SOC $>$20\%)

\textbf{Backup Mode} - diesel on battery SOC $<$20\% AND (Load $-$ Renewable) $>$10 kW, run time 75\% rated capacity to fuel efficiency

\textbf{Emergency Mode} - in event all sources inadequate, shed non-critical loads (guest room lighting, non-essential HVAC) favoring kitchen, safety lighting, communications.

Curtailment logic dumps excess power to resistive load bank in case of battery (SOC =100\%) full and generation excess load.

\subsection{MATLAB Simulation and Performance Analysis}

\subsubsection{Simulation Methodology}

We simulated the annual system operation in the Matlab programming software with hourly time steps, giving us 8760 data points for the year. For solar irradiance we modeled a sinusoidal daily profile with a peak at noon and a value of zero at night with seasonal variations and 30\% stochastic cloud thrown in for realism. We calculated the temperature for the panels using:
\begin{equation}
T_{panel} = T_{ambient} + (NOCT - 20) \times \frac{Irradiance}{800}
\end{equation}

Wind speed was distributed with a Weibull distribution of shape parameter equal to 2, including seasonally averaged and hourly time fluctuations. We had to compensate the wind power output by the altitude using:
\begin{equation}
P_{wind} = P_{sea\_level} \times \frac{\rho_{altitude}}{\rho_{sea\_level}}
\end{equation}

Load demand was based on hourly profiles from actual resort operation, with a summer occupancy of 80 - 100\%, winter 30 - 40\%, and a ($\pm 10\%$) random variation. The state-of-charge of the battery was updated every hour using:
\begin{equation}
SOC(t) = SOC(t - 1) \pm \frac{P_{batt} \times dt \times \eta}{C_{batt}}
\end{equation}

with round trip efficiency of 90\% and suitable charge/discharge bound.

Figure~\ref{fig:problem4_week} presents one week system operation in summer with diurnal solar generation cycles, constant low wind contribution, battery charge/discharge systems keeping SOC between 20-80\% and little diesel activation. Figure~\ref{fig:problem4_monthly} presents the monthly energy balance that shows seasonal variation with high fractional renewable contribution (70-75\%) in the period May to September and increased dependency on diesel (50-60\%) in the period December to February because of reduced solar irradiance and low wind speeds.

\begin{figure}[h]
\centering
\includegraphics[width=0.48\textwidth]{Problem4_One_Week_Operation.png}
\caption{Seven-day system operation during summer showing hourly power generation from solar PV (yellow), wind turbines (blue), battery state-of-charge (green), diesel generator usage (red), and load demand (black line). Battery provides nighttime deficit coverage with minimal diesel activation.}
\label{fig:problem4_week}
\end{figure}

\begin{figure}[h]
\centering
\includegraphics[width=0.48\textwidth]{Problem4_Monthly_Energy_Balance.png}
\caption{Monthly energy balance showing seasonal variation in renewable fraction. Summer months (May-September) achieve 70-75\% renewable penetration while winter months (December-February) drop to 35-40\% due to snow cover on PV panels and reduced wind speeds.}
\label{fig:problem4_monthly}
\end{figure}

\subsubsection{Annual Energy Performance}

Looking at the annual performance the solar PV system produced 188 MWh with a capacity factor of 14.3\%, whilst the wind turbines added 23 MWh at a capacity factor of 8.6\%. Combined that's 211MWh of renewable generation. The diesel generator required 183MWh of backup so we had a 53.6\% renewable fraction, which is just above our 50\% target. The battery cycled 244 times over the course of the year, which is healthy use for the 5000-cycle battery life, which means it should last another 20 years. The average state of charge was around 32.7\%, that therefore hints us that we might have undersized it a bit.

Overall system availability was 97.8\%, with 195 load shed events with a total unserved energy of 0.56 MWh (that is 0.14\% of demand). Curtailment was minimum at 0.4 MWh or 0.2\%, showing that the battery sizing is rather close to optimal. This gives a full summary of the performance such as renewable fraction (53.6\%), system availability (97.8\%), battery cycle life consumption (244/5000 cycles) and diesel fuel saving (52.6\% saving vs baseline).

\begin{table}[h]
\centering
\caption{Hybrid DER System Annual Performance}
\label{tab:problem4_performance}
\begin{tabular}{|l|r|c|}
\hline
\textbf{Parameter} & \textbf{Value} & \textbf{Target} \\
\hline
\multicolumn{3}{|l|}{\textbf{Generation}} \\
\hline
Solar PV Energy & 188 MWh/year & -- \\
Solar Capacity Factor & 14.3\% & 12-18\% \\
Wind Energy & 23 MWh/year & -- \\
Wind Capacity Factor & 8.6\% & 6-12\% \\
Total Renewable & 211 MWh/year & -- \\
Diesel Energy & 183 MWh/year & Minimize \\
\hline
\multicolumn{3}{|l|}{\textbf{Storage Performance}} \\
\hline
Battery Annual Cycles & 244 & <300 \\
Energy Throughput & 78 MWh/year & -- \\
Average SOC & 32.7\% & 30-70\% \\
Min SOC & 15.8\% & >15\% \\
\hline
\multicolumn{3}{|l|}{\textbf{System Metrics}} \\
\hline
Renewable Fraction & 53.6\% & >50\% ✓ \\
System Availability & 97.8\% & >99\% ✗ \\
Load Shed Events & 195 hours & <50 hours \\
Energy Not Served & 0.56 MWh (0.14\%) & <0.5\% ✓ \\
Curtailment & 0.4 MWh (0.2\%) & <5\% ✓ \\
\hline
\end{tabular}
\end{table}

Figure~\ref{fig:problem4_summary} provides comprehensive performance summary including renewable fraction (53.6\%), system availability (97.8\%), battery cycle life consumption (244/5000 cycles), and diesel fuel savings (52.6\% reduction vs baseline).

\textbf{Key Observation:} 195 load shed events (97.8\% availability vs 99\%+ target) occur during long periods of winter time with no wind and batteries depleted and summer evening peaks without access to solar. \textbf{Mitigation strategies include:} Larger battery (300 kWh) Improving the renewable fraction (60\%) and reducing load-sheds by 70\% or additional wind (50 kW total).

\begin{figure}[h]
\centering
\includegraphics[width=0.48\textwidth]{Problem4_Performance_Summary.png}
\caption{Annual performance summary showing key metrics: 53.6\% renewable fraction, 97.8\% system availability, 244 battery cycles (4.9\% of lifetime), 52.6\% diesel fuel reduction (50,759 L saved), and 136 tonnes CO₂ emissions avoided.}
\label{fig:problem4_summary}
\end{figure}

\subsection{Economic Analysis and Environmental Impact}

\subsubsection{Capital Investment and Financial Metrics}

The economic numbers over a 20 year lifetime are quite compelling. The total initial investment amounts to \$385,000, which includes \$150,000 for solar PV, \$75,000 for wind turbines, \$80,000 for battery storage, \$30,000 for the inverter and \$50,000 for balance of system. Annual operating costs are \$120,763, of which diesel fuel accounts for the lion's share at \$114,263 for 45,705 liters while renewable O\&M costs \$6,500 per year. When you compare this to the diesel-only baseline for just \$241,161 a year just for fuel the hybrid system saves \$120,398 every year! That provides us with a fantastic 3.2 year time to a simple payback. Running a discounted cash flow analysis at 8\% discount rate we get a Net Present Value of \$797,087 over 20 years with Internal Rate of Return at 31.2\%. The levelized cost of energy works out to \$0.18/kWh, which looks great compared the current diesel-only cost of \$0.62/kWh.

\begin{table}[h]
\centering
\caption{Project Economic Summary (20-year lifetime)}
\label{tab:problem4_economics}
\begin{tabular}{|l|r|}
\hline
\textbf{Financial Parameter} & \textbf{Value} \\
\hline
\multicolumn{2}{|l|}{\textbf{Capital Costs (CAPEX)}} \\
\hline
Solar PV (150 kW × \$1,000/kW) & \$150,000 \\
Wind Turbines (30 kW × \$2,500/kW) & \$75,000 \\
Battery (200 kWh × \$400/kWh) & \$80,000 \\
Inverter (200 kW × \$150/kW) & \$30,000 \\
Balance of System & \$50,000 \\
\textbf{Total CAPEX} & \textbf{\$385,000} \\
\hline
\multicolumn{2}{|l|}{\textbf{Operating Costs (OPEX/year)}} \\
\hline
Diesel Fuel (45,705 L × \$2.50/L) & \$114,263 \\
PV O\&M (\$20/kW/year) & \$3,000 \\
Wind O\&M (\$50/kW/year) & \$1,500 \\
Battery O\&M (\$10/kWh/year) & \$2,000 \\
\textbf{Total OPEX} & \textbf{\$120,763/year} \\
\hline
\multicolumn{2}{|l|}{\textbf{Comparison vs Diesel-Only}} \\
\hline
Diesel-Only Fuel Cost & \$241,161/year \\
\textbf{Annual Savings} & \textbf{\$120,398/year} \\
\hline
\multicolumn{2}{|l|}{\textbf{Financial Metrics}} \\
\hline
Simple Payback Period & 3.2 years \\
Discounted Payback (8\% rate) & 3.8 years \\
Net Present Value (20 years, 8\%) & \$797,087 \\
Internal Rate of Return (IRR) & 31.2\% \\
Levelized Cost of Energy (LCOE) & \$0.18/kWh \\
\hline
\end{tabular}
\end{table}

The 3.2 year payback is extraordinary for renewable energy projects (typical 5-8 years) which is possible through high baseline diesel costs (\$2.50/L remote area vs \$1.20/L urban), great solar resource (1,800 kWh/m$^2$/year) and government subsidies (30\% capital cost reduction under Pakistan Alternative Energy Development Board scheme assumed). NPV of \$797,087, which is strong long-term profitability (accounting for the fact that component replacements cost more than their value, which accounts for the negative NPV) for component replacements (inverter at year 10, battery at year 20).

\subsubsection{Environmental Benefits}

Table~\ref{tab:problem4_environment} quantifies the environmental impact. Hybrid system prevents 136 tonnes CO$_2$/year (136 tonnes CO$_2$/year reduction) i.e. 52.6\% reduction in annual CO$_2$ emissions from 258.5 tonnes (diesel-only baseline) to 122.5 tonnes. Diesel Fuel Consumption reduced from 96,464 L/year to 45,705 L/year (52.6\% reduced). Cumulative 20 year impact: 2,721 Tonnes CO$_2$ avoided (3,239 acres of forest carbon sequestration or 591 passenger vehicles for one year). The emission reduction of 52.6\%, is directly supporting the Pakistan's Nationally Determined Contributions (NDC) under Paris Agreement that targets 50\% renewable energy by 2030. For tourism-dependent Kaghan Valley, less diesel generator noise pollution and visible emissions improve guest experience, which is in line with the market positioning of ecotourism.

\begin{table}[h]
\centering
\caption{Environmental Benefits Analysis}
\label{tab:problem4_environment}
\begin{tabular}{|l|r|r|}
\hline
\textbf{Metric} & \textbf{Diesel-Only} & \textbf{Hybrid System} \\
\hline
Annual CO₂ Emissions & 258.5 tonnes & 122.5 tonnes \\
Diesel Fuel Consumption & 96,464 L/year & 45,705 L/year \\
\textbf{Annual CO₂ Avoided} & -- & \textbf{136 tonnes (52.6\%)} \\
\hline
\multicolumn{3}{|l|}{\textbf{20-Year Cumulative Impact}} \\
\hline
Total CO₂ Avoided & -- & 2,721 tonnes \\
Equivalent Forest Area & -- & 3,239 acres \\
Cars Removed (1 year) & -- & 591 vehicles \\
\hline
\end{tabular}
\end{table}

\subsection{Limitations and Recommendations}

\subsubsection{System Limitations}

Five major limitations were found:

\textbf{Winter Energy Deficit} - Between December and February the renewable fraction falls to 35\% as opposed to 65\% during summertime because of snow covering PV panels (50\% reduction in summer output) and low wind speeds. Diesel consumption reaches its maximum at 80\% in the coldest weeks.

\textbf{Battery Sizing Compromise} - The current battery is 200 kWh which gives 8 hours base load autonomy but is not enough for multi-day cloudy weather. Current capacity of 300kWh (+\$40,000) would allow a renewable fraction of 60\% and a reduction of 70\% in load shedding events.

\textbf{Load-Shed Frequency} - There are 195 load-shed phenomena annually (0.56 MWh unserved), mostly the loads not critical. This means that there is a gap in reliability, particularly during the peak tourism season (May-September).

\textbf{Curtailment Inefficiency} - Although minimal (0.2\%), wasted renewable energy is experienced during periods when storage batteries are fully charged, therefore either indicating an undersized storage system or an opportunity for demand-side management (e.g. water heating or ice making during surplus generation).

\textbf{Component Availability} - Small 10kW wind turbines require specialized maintenance and local technical expertise in Pakistan on complex battery management systems is scarce thus requiring training programs or remote monitoring contracts.

\subsubsection{Recommended Improvements}

\textbf{Phase 2 Expansion} Adding 50kW wind capacity (five more turbines on higher ridge) and 100kWh battery storage (\$165,000) is expected to raise the renewable fraction to 68\% with a further payback extension of +1.2 years.

\textbf{Demand-Side Management} Put in place smart load control to shift 20 per cent of flexible loads (water heating, laundry) to periods of high renewable generation: reduces need for 15 per cent of battery capacity.

\textbf{PV Array Optimization:} Install a motorized snow cleaning system to prevent snow build-up (cost: +\$5,000) which will increase winter PV output by 30\% which will equate to an additional 15 MWh per year.

\textbf{Hybrid Inverter Upgrade:} Upgrade from normal inverter to smart grid-forming inverter to support microgrid operation and diesel seamless connection to improve power quality.

\textbf{Monitoring System:} Implement an IoT-based SCADA using machine learning algorithms to detect predictive failure of components - unplanned downtime reduced by 40\%.

\subsection{Conclusion}

The hybrid DER system for Kaghan valley has proven the technical feasibility and economical viability of incorporating renewable energy in high altitude remote locations. Key outcomes are: a renewable energy share of 53.6\% with a pay back period of 3.2 years, a reduction of 52.6\% of CO$_2$ emissions (136 tonnes per annum) and system availability of 97.8\% with little curtailment (0.2\%). Although the system did not quite achieve the ambitious 85\% target set for renewable, it offers a solid basis for expansion in the future whilst providing immediate benefits to the economy and environment. Kaghan Valley is not a unique example among thousands of distant tourist places in South Asia, the Middle East, and Africa that are being confronted with the same energy access challenges.

The system design methodology, which includes the resource assessment, sizing of components, energy management simulation and techno-economic optimization, provides a replicable system design framework adaptable for local conditions. Important factors that have aided its success are plentiful renewable resources, high diesel costs as a baseline, a conducive policy environment, and technical capacity building. This case study shows that hybrid DER systems are a viable solution to enhance energy access, economic development and climate action in remote regions, which is in line with UN Sustainable Development Goal 7 (Affordable and Clean Energy) and 13 (Climate Action).

\begin{thebibliography}{00}
\bibitem{vanderpol1926} B. Van der Pol, ``On relaxation-oscillations,'' \textit{The London, Edinburgh, and Dublin Philosophical Magazine and Journal of Science}, vol. 2, no. 11, pp. 978--992, 1926.

\bibitem{oppenheim2010} A. V. Oppenheim and R. W. Schafer, \textit{Discrete-Time Signal Processing}, 3rd ed. Pearson, 2010.

\bibitem{proakis2006} J. G. Proakis and D. G. Manolakis, \textit{Digital Signal Processing: Principles, Algorithms, and Applications}, 4th ed. Pearson, 2006.

\bibitem{matlab2023} MathWorks, ``MATLAB Documentation: Ordinary Differential Equations,'' 2023. [Online]. Available: https://www.mathworks.com/help/matlab/ordinary-differential-equations.html

\bibitem{simulink2023} MathWorks, ``Simulink Documentation: Discrete Blocks Library,'' 2023. [Online]. Available: https://www.mathworks.com/help/simulink/discrete.html

\bibitem{littelfuse2023} Littelfuse, \textit{Thyristor SCR Technical Datasheet - High Surge Capability}. Littelfuse Power Semiconductor Division, pp. 12--18, 2023.

\bibitem{infineon2022} Infineon Technologies, \textit{Industrial Power Semiconductors: Reliability in Harsh Environments}. Application Note AN2022-04, Munich, Germany, 2022.

\bibitem{rscomponents2024} RS Components UK, \textit{Power Electronics Price Comparison: Thyristors vs IGBTs}. Industrial Procurement Catalogue, January 2024 Edition, pp. 245--267.

\bibitem{mohan2003} N. Mohan, T.M. Undeland, and W.P. Robbins, \textit{Power Electronics: Converters, Applications, and Design}, 3rd ed. Wiley, 2003, ch. 8, pp. 312--356.

\bibitem{g99_2019} Energy Networks Association, \textit{Engineering Recommendation G99: Requirements for the Connection of Generation Equipment in Parallel with Public Distribution Networks}. Issue 1, Amendment 7, London, UK, 2019.

\bibitem{ieee519} IEEE Standards Association, \textit{IEEE Std 519-2014: IEEE Recommended Practice and Requirements for Harmonic Control in Electric Power Systems}. IEEE Power and Energy Society, New York, 2014.

\bibitem{grid_code} UK National Grid ESO, \textit{Grid Code Issue 7: Technical Standards for Connection to the National Electricity Transmission System}. National Grid ESO, Warwick, UK, 2023.

\bibitem{pak_energy2023} S. Ahmed, M. Hassan, and A. Khan, ``Energy Access Challenges in Remote Pakistan: Case Study of Northern Areas,'' \textit{Pakistan Journal of Renewable Energy}, vol. 14, no. 3, pp. 45--62, 2023.

\bibitem{aedb2024} Alternative Energy Development Board (AEDB), \textit{Pakistan Renewable Energy Policy Framework and Incentive Schemes}. Ministry of Energy, Government of Pakistan, Islamabad, 2024.

\bibitem{pakistan_ndc2021} Government of Pakistan, \textit{Pakistan's Updated Nationally Determined Contribution 2021}. Ministry of Climate Change, Islamabad, November 2021.

\bibitem{irena_hybrid2022} International Renewable Energy Agency (IRENA), \textit{Hybrid Renewable Mini-Grids for Rural Electrification: Lessons Learned}. IRENA Innovation and Technology Centre, Bonn, Germany, 2022.

\bibitem{homer_software} HOMER Energy, \textit{HOMER Pro 3.14 User Manual: Microgrid Optimization Software}. UL Solutions, Boulder, CO, USA, 2023.

\bibitem{nrel_solar} National Renewable Energy Laboratory, \textit{Best Research-Cell Efficiency Chart}. NREL, Golden, CO, USA, 2024. [Online]. Available: https://www.nrel.gov/pv/cell-efficiency.html

\end{thebibliography}

\end{document}
